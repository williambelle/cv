\cvsection{Expériences Professionnelles}

\begin{cventries}

  \cventry
    {Scrum Master et Software Developer}
    {Health On the Net Foundation (HON)}
    {Depuis 2015}
    {18 mois}
    {
      \begin{cvitems}
        \item{Responsable des développements informatiques chez HON. En charge d'une
          équipe de 2/3 personnes.
        }
        \item{Réalisation de l'application MedAMGe pour
          {\color{awesome-skyblue}
            \href{https://itunes.apple.com/us/app/medamge/id469762154?mt=8}{iOS}},
          {\color{awesome-skyblue}
            \href{https://play.google.com/store/apps/details?id=org.healthonnet.medamge.android&hl=en}{Android}}
          ainsi qu'une version
          {\color{awesome-skyblue}
            \href{https://www.medamge.ch}{web}}
          . MedAMGe est un \\annuaire (genevois) qui permet de rapidement localisé
          un professionnel de la santé.
        }
        \item{Réalisation d'un quiz sur l'anatomie et la physiologie du corps
          humain (3D Anatomy Quiz). \\L'application est disponible pour le
          {\color{awesome-skyblue}
            \href{https://3danatomyquiz.kaahe.org}{web}},
          {\color{awesome-skyblue}
            \href{https://itunes.apple.com/us/app/t-rf-ly-jsdk/id1020122013?mt=8}{iOS}}
          et
          {\color{awesome-skyblue}
            \href{https://play.google.com/store/apps/details?id=org.kaahe.anatomyQuiz&hl=en}{Android}}.
        }
        \item{Mise en place d'une plateforme de paiement (via API Paypal)
          pour la certification HONcode.
        }
      \end{cvitems}
    }

  \cventry
    {Software Developer}
    {Health On the Net Foundation (HON)}
    {Novembre 2010 - 2014}
    {4 ans}
    {
      \begin{cvitems}
        \item{Développement du portail web
          {\color{awesome-skyblue}\href{http://www.kaahe.org/}{KAAHE}}
          dont la mission est de promouvoir des informations fiables \\
          sur la santé pour la communauté Arabe. Actuellement plus de 1.6
          million d'utilisateur avec tout ce \\que cela implique en termes de
          sécurité monitoring, backup et configuration serveur.
        }
        \item{Développement de l'application Health Encyclopedia pour
          {\color{awesome-skyblue}
            \href{https://itunes.apple.com/us/app/almwsw-t-alshyt/id624559765?l=fr&ls=1&mt=8}{iOS}}
          et
          {\color{awesome-skyblue}
            \href{https://play.google.com/store/apps/details?id=org.kaahe.kaaheApp&hl=en}{Android}}
          . Une application gratuite \\fournissant un accès à tous les sujets
          disponibles sur le site web mais en version "offline".
        }
        \item{Développement d'une application de gestion du contenu pour le
          site web KAAHE. Réalisation d'un \\workflow spécifique (rôle, gestion
          de version du contenu, traduction).
        }
        \item{Réalisation d'un Moteur de recherche d'informations médicales -
          {\color{awesome-skyblue}
            \href{http://everyone.khresmoi.eu/}{Khresmoi For Everyone}}
            - avec une \\implication à tous les niveaux du projet (robots,
            indexation, moteur de recherche et interface).
        }
        \item{Développement d'une extension Firefox pour le HONcode -
          {\color{awesome-skyblue}
          \href{https://addons.mozilla.org/en-US/firefox/addon/hon-toolbar/}{HON Toolbar}}.
        }
      \end{cvitems}
    }

  \cventry
    {Stagiaire - Web Developer}
    {Health On the Net Foundation (HON)}
    {Mai - Octobre 2010}
    {6 mois}
    {
      \begin{cvitems}
        \item{Développement du portail web pour le
          {\color{awesome-skyblue}\href{http://scaht.org/}{SCAHT}}.
        }
        \item{Développement et maintenance de composants pour la gestion du
          {\color{awesome-skyblue}
            \href{http://www.hon.ch/HONcode/Pro/Visitor/visitor.html}{HONcode}}.
        }
      \end{cvitems}
    }

  \cventry
    {Stagiaire - Assistance et Conseils Informatique}
    {Services Industriels de Genève (SIG)}
    {Mars - Juin 2004}
    {4 mois}
    {
      \begin{cvitems}
        \item{Gestion du parc micro-informatique, installation du matériel,
          montage de serveur.
        }
      \end{cvitems}
    }

\end{cventries}
